\section{Observations}

The dataset was retrieved from graphs for \cite{bouwens14} and \cite{mclure09} 
using DEXTER
\url{http://dc.zah.uni-heidelberg.de/dexter/ui/ui/custom}


\begin{figure}
\epsscale{1.00}
\plotone{fig/observational_data.pdf}
\caption{Observational data from \cite{bouwens14,mclure09}and \cite{willott13}.}
\label{graph_observational_data}
\end{figure}

  \subsection{CFHTLS}
  
  CFHTLS - Canada-France-Hawaii Telescope Legacy Survey - 
MegaCam\citep{willott13}

  ``We identify a sample of 40 Lyman break galaxies brighter than
    magnitude z ′ = 25.3 across an area of almost 4 square degrees.
    Sensitive spectroscopic observations of seven galaxies provide 
    redshifts for four, of which only two have moderate to strong 
    Lyα emission lines. All four have clear continuum breaks in 
    their spectra.'' \cite{willott13}.

  ``The imaging data used to select high-redshift galax-
    ies come primarily from the 3.6 m Canada-France-Hawaii
    Telescope. Optical observations with MegaCam in the
    u ∗ g ′ r ′ i ′ z ′ filters are from CFHTLS Deep which covered
    four ≈ 1 square degree fields with typical total integra-
    tion time of 75 ks in u ∗ , 85 ks in g ′ , 145 ks in r ′ , 230 ks in
    i ′ and 175 ks in z ′ . The seeing in the final stacks at i ′ and
    z ′ range from 0.66 to 0.76 arcsec. The data used here are
    from the 6th data release, T0006, which contains all the
    data acquired over the five years of the project '' \cite{willott13}.

  ``These optical data are complemented by near-IR data
    from the WIRCam Deep Survey (WIRDS; Bielby et al.
    2012). WIRDS used the WIRCam near-IR imager at
    the CFHT'' \cite{willott13}.

   About the Luminosity function: sect 6.3
   ``Because our LBG sample covers only a limited range of
    apparent and absolute magnitudes, we cannot use it to determine
    the full galaxy luminosity function at z = 6. The luminosity
    function at the break and at fainter magnitudes has already
    been well studied from deep HST surveys over small sky areas
    (Bouwens et al. 2007). The main contribution of our study is
    to determine the space density of very rare, highly luminous
    LBGs.''\cite{willott13}.

  \subsection{HUDF09}

  ``Table 1 summarizes the search fields used for the z ∼ 5-
    8 LF determinations and the approximate depths of the
    available ACS+WFC3/IR observations. Our primary
    data set consists of the full two-year WFC3/IR obser-
    vations of the HUDF and two flanking fields obtained
    with the 192-orbit HUDF09 program (PI Illingworth:
    GO 11563). Our second data set is the ∼145 arcmin 2
    ACS+WFC3/IR observations over the wide-area Early
    Release Science (Windhorst et al. 2011) and CDF-South
    CANDELS'' \cite{bouwens12}
    
  ``A detailed summary of the ACS HUDF, HUDF-Ps, and GOODS
    data we use for our dropout selections is provided in our previous
    work ( B06). Nevertheless, a brief description of the data is in-
    cluded here. The ACS HUDF data we use are the version 1.0 re-
    ductions of Beckwith et al. (2006) and extend to 5  point-source
    limits of 29Y30 in the B 435 V 606 i 775 z 850 bands. The HUDF-Ps re-
    ductions we use are from B06 and take advantage of the deep
    (k72 orbit) BViz ACS data fields taken in parallel with the HUDF
    NICMOS program (Thompson et al. 2005). Together the parallel
    data from this program sum to create two very deep ACS fields
    that we can use for dropout searches. While of somewhat var-
    iable depths, the central portions of these fields (12Y20 arcmin 2 )
    reach some 0.6Y0.9 mag deeper than the data in the original ACS
    GOODS program (Giavalisco et al. 2004a). Finally, for the ACS
    GOODS reductions, we use an updated version of those generated
    for our previous z 6 study (B06).'' \cite{bouwens06}

  \subsection{UKIDSS \& SDXS}
    ``The UDS is the deepest of five near-IR surveys currently underway
      at the UK InfraRed Telescope (UKIRT) with the new WFCAM
      imager (Casali et al. 2007) which together comprise the UKIDSS
      (Lawrence et al. 2007). The UDS covers an area of 0.8 deg 2 cen-
      tred on RA = 02:17:48, Dec. = −05:05:57 (J2000) and is already
      the deepest, large area, near-IR survey ever undertaken. The data
      utilized in this paper were taken from the first UKIDSS Data Re-
      lease (DR1; Warren et al. 2007), which included JK imaging of the
      entire UDS field to 5σ depths of J = 23.9, K = 23.8 (1.6 arcsec
      diameter apertures). The UKIDSS DR1 became publicly available
      to the world wide astronomical community in 2008 January and can
      be downloaded from the WFCAM Science Archive. 2
      The UDS field is covered by a wide variety of deep, multiwave-
      length observations ranging from the X-ray through to the radio (see
      Cirasuolo et al. 2008 for a recent summary). However, for this study
      the most important multiwavelength observations are the deep op-
      tical imaging of the field taken with Suprime-Cam (Miyazaki et al.
      2002) on Subaru as part of the Subaru/XMM–Newton Deep Survey
      (Sekiguchi et al. 2005). The optical imaging consists of five over-
      lapping Suprime-Cam pointings, and covers an area of 1.3 deg 2 .
      The whole field has been imaged in the BV Ri z filters, to typical
      5σ depths of B = 27.9, V = 27.4, R = 27.2, i = 27.2 and z = 26.2
      (1.6 arcsec diameter apertures). The reduced optical imaging of the
      SXDS is now publicly available 3 and full details of the observations,
      data reduction and calibration procedures are provided in Furusawa
      et al. (2008). The high-redshift galaxies investigated in this study
      were selected from a contiguous area of 0.63 deg 2 (excluding areas
      contaminated by bright stars and CCD blooming) covered by both
      the UDS near-IR and SXDS optical imaging.'' \cite{mclure09}
      
\subsection{The Drop-out Technique - Lyman Break Technique}

\cite{steidel03}