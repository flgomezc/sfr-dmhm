\section{Observations}

This paper is based on four main observational data sets. The most recent is
from the Hubble Space Telescope Legacy plus ground telescopes \citep{bouwens14}.
The second set is from the Canada-France Hawaii Telescope\citep{willott13} and
the last one by the UK Infrared Telescope and the Subaru Telescope
\cite{mclure09}. Those observations where made using the Drop-out technique.

The data from the HSTL is a compilation of previous works since
2006 \citep{bouwens06}, wich includes also observations after the 2009
upgrade mission\citep{bouwens12}. They estimate the uncertainty in their LF due
to cosmic variance. Taking account they used five independent surveys, they get
jus $\sigma\sim 10\%$\citep{bouwens14} using the Cosmic Variance Calculator
developed by Trenti \& Stiavelli (2008).



The dataset was retrieved from graphs for \cite{bouwens14} and \cite{mclure09} 
using
GAVO-DEXTER\footnote{\url{http://dc.zah.uni-heidelberg.de/dexter/ui/ui/custom}}.




Willott et al. \cite{willott13} presented the sixth release of the Canada-France-Hawaii Telescope Legacy Survey CFHTLS. The observations where performed over four separated fields covering a total area $\sim 4 \deg^2$ (a large area), it gives this survey great robustness.
Optical observations used MegaCam with $u^* g' r' i' z'$ filters. The main selection criteria: all the objects must be brighter than magnitude $z' = 25.3$. The final number of LBGs founded was 40. Moreover, they get spectroscopic confirmation for 7 candidates using GMOS spectrograph on the Gemini Telescopes, which has a field of view $\ll 1 \deg^2$. They show incompleteness in the sample due to foreground contamination and the detection algorithm; there is no warranty to have every object brighter than the limit magnitude on the faint limit. The full galaxy LF at $z=6$ cannot be obtained as in other studies. Nevertheless, this survey was focussed on the highly luminous LBGs. LF is calculated using the stepwise maximum likelihood method of Efstathiu et al. [Cite required!!!], within magnitudes from $M_{1350} = -22.5$ up to $-20.5$. The luminosity function of $z=5.9$ shows an exponential decline at the bright end, where feedback processes and inefficient last cooling limites star forming in bright galaxies hosted in the most massive halos.






\begin{figure}
\epsscale{1.00} 
\plotone{fig/observational_data.pdf}
\caption{Observational data from \cite{bouwens14,mclure09}and \cite{willott13}.}
\label{graph_observational_data}
\end{figure}

     
\subsection{The Drop-out Technique - Lyman Break Technique}

\cite{steidel03}