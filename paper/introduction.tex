\section{Introduction}
Cosmological simulations still playing an important role on astrophysics and
cosmology. They have become the bridge between observations and theory, a
laboratory where is possible to create universes with different parameters and
new physics.

This is an important tool in the study of dark matter. Since its existence was
purposed, DM has been not observed directly neither on telescopes or high energy
particle colliders, there exists many other theories explaining the observed
universe behavior modifying the gravity itself.

The most successful and accepted cosmological model is the Lambda-CDM. 
Dark matter is a significative component of the universe.
\citep{trimble87}

Hierarchy Structure Evolution: Early formation of small structures
merging on major structures after.

From simulations, Halo Mass Function as function of redshift (or time).


In this paper is established a relation between dark matter and galaxy
luminosity functions at high redshifrt. This relation is a direct
connection between the Lambda-CDM model and the farthest galaxy observations
made with space and grounded telescopes.


Star formation rate as function of time. Peak at $z\sim 2$.


``An important frontier in the study of very high redshift
galaxies remains the study of their stellar populations.
Galaxies within a few hundred million years of the Big
Bang are expected to be quite different from galaxies at
lower redshift, with significantly younger ages and lower
metallicities. For very young and chemically immature
systems, changes in the stellar population could include a
transition to a more top-heavy IMF (e.g., Bromm \& Lar-
son 2004), evolution in the dust composition (e.g., due to
changes in the dust production mechanism: Maiolino et
al. 2004), as well as a much lower dust extinction overall
(e.g., Bouwens et al. 2009; Finlator et al. 2011; Dayal \&
Ferrara 2012).'' BOWENS 2014 BETA SLOPE




Main Objetive: Reproduce the observed luminosity function at redshift $z=5.9$
from a DMH catalog from simulations. 

This paper is organized as follows: In section 2... in section 3...

  \subsection{Halo Mass Function (HMF)}


  
%  The HMF is created using the DMH catalog from the Multidark Database 
%\citep{riebe13}.
  
  
  \subsection{Cosmic Variance}
  
%  Is important to know and understand the existence of local fluctuations   or 
%inhomogeneities due the observed scale.  For this part is necesary to study the 
%HMF on smaller boxes

``One of the most fundamental properties of galaxy sub-
populations at any epoch is their number density. However,
observational estimates of galaxy number densities in finite
volumes are subject to uncertainty due to cosmic variance,
arising from underlying large-scale density fluctuations and
leading to uncertainties in excess of naïve Poisson errors.
Note that this source of uncertainty is referred to as ‘sample
variance’ in other branches of cosmology. For sampling vol-
umes much larger than the typical clustering scale of the ob-
served objects, cosmic variance is not significant.''A COSMIC VARIANCE COOKBOOK

``However many important existing surveys have a sampling volume that
is small enough that cosmic variance may dominate the un-
certainties. This may be particularly true at high redshift,
where galaxies are expected to be much more strongly clus-
tered than dark matter (Kauffmann et al. 1999; Baugh et al.
1999; Coil et al. 2004; Moster et al. 2009). Still, many pub-
lished quantities which are based on number density (e.g. lu-
minosity functions, stellar mass functions, etc.) are quoted
with error budgets that do not properly account for cosmic
variance. As shown by Trenti \& Stiavelli (2008), the normal-
ization and the slope of high-redshift luminosity functions can
be affected by cosmic variance errors.''A COSMIC VARIANCE COOKBOOK

``Somerville et al. (2004) provided predictions that could be
used to estimate cosmic variance as a function of mean red-
shift and survey volume, using the number density of the pop-
ulation to estimate the bias, assuming one galaxy per halo.'' A COSMIC VARIANCE
COOKBOOK


===========>>>>>>>>> COSMIC VARIANCE CALCULATOR <<<<<<<<<===============
Results obtained using CosmicVarianceCalculator v1.02 
Developed by Michele Trenti \& Massimo Stiavelli 
If you use these results in scientific papers, please refer to:  
Trenti \& Stiavelli (2008), ApJ, 676, 767 


\section{Linking Galaxy Luminosity Function (GLF) and
  Dark Matter Halo (DMH) mass}

READ THIS PAPER
Vale - Ostriker
arXiv:astro-ph/0402500v2 6 Jul 2004
Mon. Not. R. Astron. Soc.
Linking halo mass to galaxy luminosity

``In recent years, N-body numerical simulations have given us
a good understanding of dark matter structure for standard
cosmological scenarios, while large scale observational
surveys have done the same for the distribution of galaxies'' arXiv:0402500v2

``More indirect approaches have also been studied.
The halo occupation distribution (HOD) model (Seljak
2000; Benson 2001; Bullock, Wechsler, \& Somerville 2002;
Zheng et al. 2002; Berlind \& Weinberg 2002; Berlind et al.
2003; Magliocchetti \& Porciani 2003) is based on the prob-
ability P(N|M) that a halo of mass M is host to N galax-
ies. By specifying the P(N|M) function, along with some
form for the distribution of dark matter and galaxies within
each halo, it is then possible to relate different statistical
indicators of the dark matter and galaxy distributions, such
as correlation functions, to each other. This fully specifies
the bias between the galaxy and the underlying matter dis-
tributions. '' arXiv:0402500v2


``Other authors have used a slightly different method.
Instead of trying to specify the number of galaxies in each
halo, they treat the halo as a whole and identify it with
a galaxy group. Then, by comparing the group luminosity
function with the halo mass function, they obtain the lu-
minosity associated with each halo (Peacock \& Smith 2000;
Marinoni \& Hudson 2002), '' arXiv:0402500v2

============>>>>>>>>> Check Peacock \& Smith 2000; Marioni \& Hudson 202





%  In one hand we have observational results: the GLF for star-forming galaxies 
%at high redshift (z=5.9)\citep{bouwens06,willott13}  expressed in terms of 
%magnitude in the ultraviolet range (M1350). In the other hand we have a DMH  
%atalog, result from the Multidark Simulation with Plank Cosmology. 
%\citep{Multidark}

 % A $1000 \textrm{ Mpc} ^3 \textrm{h}^{−3}$ cubic box containing near to  
%$11\times10^6$ dark matter halos.

%  The idea is to connect the observed GLF with the DMH catalog. We assume that  
%each halo contains one and just one galaxy, and its luminosity is given by:
%  \begin{equation}
%  L_\textrm{galaxy}=\alpha M_\textrm{halo}^\beta 
%  \end{equation}
%but this function is given in magnitude units. Is necesary to convert 
%magnitude to luminosity.

\subsection{Magnitude to Luminosity}

Luminosity Functions (LF) are usually expressed in terms of magnitude instead 
luminosity. Luminosity is the energy emmited by a source in a given wavelengt range, 
is a physical quantity. 
%The luminosity 
%$L_\nu$ at a given frecuency $\nu$ has $[\textrm{W}\textrm{Hz}^{-1}]$ or 
%$[\textrm{erg}\textrm{s}^{-1}\textrm{Hz}^{-1}]$ units.
Magnitude is a classification inherited from ancient Greeks, this
quantifies the response of the first astrometric device: the human eye, this
perception grows loarithmically with the retrieved radiation.

Luminosity of any object can be compared with Sun Luminosity $(L_{\lambda
\odot})$ at any wavelengt. With the Sun Magnitude as reference
$(M_{\lambda \odot})$, the absolute magnitud of the objet  at an specific
wavelengt is given by: 
 \[ M_{\lambda} = M_{\lambda \odot} - 2.5 \log_{10}\left( 
\frac{L_\lambda}{L_{\lambda \odot}} \right) \]
  The solar absolute magnitude in the U filter is $M_{U\odot} = 5.61$,
and the solar luminosity in the same filter is $L_{U\odot} = 10^{18.48} 
\textrm{ergs s}^{-1}\textrm{Hz}^{-1}$ or $ L_{U\odot} = 3.02 \times 10^{18} 
\textrm{ergs s}^{-1}\textrm{Hz}^{-1}$. The solar luminosity can be used as
refference unit, in this fashion, the typical luminosity of a galaxy can be expresed in
terms of $10^{8}-10^{11}$ times the sun luminosity.

  The absolute magnitude of a galaxy equation in the U filter:
  %\[ M_{U} = 5.61 - 2.5 \log_{10}(L_{U}) + 2.5\times18.48\]
  %gives the magnitude in the U filter of an astrophysical source:
  \[ M_{U} = 51.82 - 2.5 \log_{10}(L_{U}) \]

 %\subsection{Best Fitting Parameters}