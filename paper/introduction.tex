\section{Introduction}

Dark matter is a significative component of the universe.

\citep{trimble87}

Hierarchy Structure Evolution: Early formation of small structures
merging on major structures after.

From simulations, Halo Mass Function as function of redshift (or time).

Star formation rate as function of time. Peak at $z\sim 2$.

Main Objetive: Reproduce the observed luminosity function at redshift $z=5.9$
from a DMH catalog from simulations. 

  \subsection{Halo Mass Function (HMF)}
  
  The HMF is created using the DMH catalog from the Multidark Database 
\citep{riebe13}.
  
  
  \subsection{Cosmic Variance}
  
  Is important to know and understand the existence of local fluctuations   or 
inhomogeneities due the observed scale.  For this part is necesary to study the 
HMF on smaller boxes

\section{Relationship between Galaxy Luminosity Function (GLF) and
  Dark Matter Halo (DMH) mass}

  In one hand we have observational results: the GLF for star-forming galaxies 
at high redshift (z=5.9)\citep{bouwens06,willott13}  expressed in terms of 
magnitude in the ultraviolet range (M1350). In the other hand we have a DMH  
catalog, result from the Multidark Simulation with Plank Cosmology. 
%\citep{Multidark}

  A $1000 \textrm{ Mpc} ^3 \textrm{h}^{−3}$ cubic box containing near to  
$11\times10^6$ dark matter halos.

  The idea is to connect the observed GLF with the DMH catalog. We assume that  
each halo contains one and just one galaxy, and its luminosity is given by:
  \begin{equation}
  L_\textrm{galaxy}=\alpha M_\textrm{halo}^\beta 
  \end{equation}
but this function is given in magnitude units. Is necesary to convert 
magnitude to luminosity.

  \subsection{Magnitude to Luminosity}
  The Luminosity is an intrinsic propertie of the stars. It doesn't deppends of 
the distance.  It's directly related to the energy flux emmited. The luminosity 
$L_\nu$ at a given frecuency $\nu$ has $[\textrm{W}\textrm{Hz}^{-1}]$ or 
$[\textrm{erg}\textrm{s}^{-1}\textrm{Hz}^{-1}]$ units.

  The magnitude is brightnes star classification inherited from ancient Greeks. 
It deppends of the stellar distance. The absolute magnitude is a modern 
classification independent of  distance. Taking the Sun as reference, the 
absolute magnitud at a given wavelengt is given by: 
 \[ M_{\lambda} = M_{\lambda \odot} - 2.5 \log_{10}\left( 
\frac{L_\lambda}{L_{\lambda \odot}} \right) \]
  The solar absolute magnitude in the U filter is $M_{U\odot} = 5.61$,
and the solar luminosity in the same filter is $L_{U\odot} = 10^{18.48} 
\textrm{ergs s}^{-1}\textrm{Hz}^{-1}$ or $ L_{U\odot} = 3.02 \times 10^{18} 
\textrm{ergs s}^{-1}\textrm{Hz}^{-1}$.

  Replacing in the absolute magnitude equation:
  \[ M_{U} = 5.61 - 2.5 \log_{10}(L_{U}) + 2.5\times18.48\]
  gives the magnitude in the U filter of an astrophysical source:
  \[ M_{U} = 51.82 - 2.5 \log_{10}(L_{U}) \]

  \subsection{Best Fitting Parameters}

