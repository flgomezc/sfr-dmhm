\documentclass[manuscript]{aastex}
%\documentclass[preprint2]{aastex}
%\usepackage{graphicx}
%\usepackage{graphics}
%\usepackage[dvips]{epsfig}
\usepackage{epstopdf}

\newcommand{\vdag}{(v)^\dagger}
\newcommand{\myemail}{skywalker@galaxy.far.far.away}

\shorttitle{A Connection Between Star Formation Rate and Dark Matter Halos at Z 
\~ 6}
\shortauthors{G\'omez-Cort\'es}

\begin{document}

\title{A Connection Between Star Formation Rate and Dark Matter Halos at 
$Z\sim6$ In 2013 Planck Cosmology.}

\author{F.L. G\'omez-Cort\'es\altaffilmark{1} }
\affil{Departamento de Física, Universidad de los Andes, Colombia}

\begin{abstract}
This work relates baryonic matter and dark matter at redshift $z=5.9$ using 
observational data from CFHTLS \citep{willott13}, HUDF09 
\citep{bouwens06,bouwens12}, UKIDSS and SDXS \citep{mclure09}, and results of 
the Multidark Simulation \citep{riebe13} in a cubic box of $1000 \textrm{Mpc 
h}^{-1}$ length with 2013 Planck Cosmology. The Luminosity Function (LF) is 
fitted via four parameters with the Markov Chain Monte Carlo method. The 
relationship between the Dark Matter Halos Mass and Star Formation Rate is 
obtained using the relationship between the UV  continuum (from the fitted LF) 
and Star Formation Rate (SFR) by \cite{kennicutt98}.Cosmic variance effects are 
studied on smaller boxes of $250 \textrm{Mpc h}^{-1}$ length.

%Luminosity Function - Dark Matter Halos, Star Formation Rate - Dark Matter 
Halos.
\end{abstract}
\keywords{Dark Matter, LF, SFR, High Redshift Galaxies, Reionization}

\section{Introduction}

Dark matter is a significative component of the universe.

\citep{trimble87}

Hierarchy Structure Evolution: Early formation of small structures
merging on major structures after.

From simulations, Halo Mass Function as function of redshift (or time).

Star formation rate as function of time. Peak at $z\sim 2$.

Main Objetive: Reproduce the observed luminosity function at redshift $z=5.9$
from a DMH catalog from simulations. 

  \subsection{Halo Mass Function (HMF)}
  
  The HMF is created using the DMH catalog from the Multidark Database 
\citep{riebe13}.
  
  
  \subsection{Cosmic Variance}
  
  Is important to know and understand the existence of local fluctuations   or 
inhomogeneities due the observed scale.  For this part is necesary to study the 
HMF on smaller boxes

\section{Relationship between Galaxy Luminosity Function (GLF) and
  Dark Matter Halo (DMH) mass}

  In one hand we have observational results: the GLF for star-forming galaxies 
at high redshift (z=5.9)\citep{bouwens06,willott13}  expressed in terms of 
magnitude in the ultraviolet range (M1350). In the other hand we have a DMH  
catalog, result from the Multidark Simulation with Plank Cosmology. 
%\citep{Multidark}

  A $1000 \textrm{ Mpc} ^3 \textrm{h}^{−3}$ cubic box containing near to  
$11\times10^6$ dark matter halos.

  The idea is to connect the observed GLF with the DMH catalog. We assume that  
each halo contains one and just one galaxy, and its luminosity is given by:
  \begin{equation}
  L_\textrm{galaxy}=\alpha M_\textrm{halo}^\beta 
  \end{equation}
but this function is given in magnitude units. Is necesary to convert 
magnitude to luminosity.

  \subsection{Magnitude to Luminosity}
  The Luminosity is an intrinsic propertie of the stars. It doesn't deppends of 
the distance.  It's directly related to the energy flux emmited. The luminosity 
$L_\nu$ at a given frecuency $\nu$ has $[\textrm{W}\textrm{Hz}^{-1}]$ or 
$[\textrm{erg}\textrm{s}^{-1}\textrm{Hz}^{-1}]$ units.

  The magnitude is brightnes star classification inherited from ancient Greeks. 
It deppends of the stellar distance. The absolute magnitude is a modern 
classification independent of  distance. Taking the Sun as reference, the 
absolute magnitud at a given wavelengt is given by: 
 \[ M_{\lambda} = M_{\lambda \odot} - 2.5 \log_{10}\left( 
\frac{L_\lambda}{L_{\lambda \odot}} \right) \]
  The solar absolute magnitude in the U filter is $M_{U\odot} = 5.61$,
and the solar luminosity in the same filter is $L_{U\odot} = 10^{18.48} 
\textrm{ergs s}^{-1}\textrm{Hz}^{-1}$ or $ L_{U\odot} = 3.02 \times 10^{18} 
\textrm{ergs s}^{-1}\textrm{Hz}^{-1}$.

  Replacing in the absolute magnitude equation:
  \[ M_{U} = 5.61 - 2.5 \log_{10}(L_{U}) + 2.5\times18.48\]
  gives the magnitude in the U filter of an astrophysical source:
  \[ M_{U} = 51.82 - 2.5 \log_{10}(L_{U}) \]

  \subsection{Best Fitting Parameters}



\section{The Fitting Model}

  This fitting model contains four parameters: $\left(m/M\right)_0$, $M_1$, 
$\beta$ and $\gamma$, 
  where $m:=$ Stellar mass, and $M :=$ Dark Matter Halo mass.
  \begin{equation}
  \frac{m}{M} = 2 \left( \frac{m}{M} \right)_{0} 
		    \left[ \left(\frac{M}{M_1}\right)^{-\beta} + 
\left(\frac{M}{M_1}\right)^{\gamma} \right]^{-1} 
  \end{equation}
  This is similar to the proposed by \cite{moster10}.
  Another model cited in the article contains five parameters: $m_0$, $M_1$, 
$\beta$, $\gamma_1$ and $\gamma_2$.
  \[ m(M) = m_0 \frac{ (M/M_1)^{\gamma_1}}{ \left[ 1 + (M/M_1)^\beta \right]^{ 
(\gamma_1-\gamma_2)/\beta}} \]


   ``We use a statistical approach to determine the relationship between the 
stellar masses of galaxies and the masses
  of the dark matter halos in which they reside. We obtain a parameterized 
stellar-to-halo mass (SHM) relation by
  populating halos and subhalos in an N-body simulation with galaxies and 
requiring that the observed stellar mass
  function be reproduced.'' \citep{moster10}

\section{Star Formation Rate}

On the study of far galaxies individual stellar spectrum is unresolved, is not 
possible to make a detailed census of the galaxy popullation. Only is possible 
to get information from the whole stellar population, an integrated spectrum.

Kennicutt pruposed a method in wich a linear relation between luminosity and 
SFR can be assumed. This model allows to estimate the young stars fraction and 
the mean SFR over periods of $10^8 - 10^9 \textrm{yr}$. The luminosity in the 
modfel, comes from the UV and the FIR broadband, also from speciffic 
recombination lines. 

In a typical galaxy spectrum the visible wavelengths are dominated by the main 
sequence stars (A to early F) and G-K giants. In few wavelength ranges we have 
a significative contribution from the young stars rather than the old stars. 
The infrared and far infrared wavelengths emission is dominated by dust, this 
dust is  heated by the whole stellar popullation, in particular by young, 
UV-bright stars \citep{law11}.The UV broadband emission is dominated by blue 
stars with temperature near to 40.000K. These hot and massive stars has a 
lifetime of $10^8\textrm{Gyr}$, they spend their nuclear fuel faster than smaller 
and cooler sunlike stars.

The relation between UV luminosity and Star Formation Rate \citep{kennicutt98} 
is given by:
  $ \textrm{SFR}\left(M_\odot \textrm{yr}^{-1}\right) 
      = 1.4 \times 10^{28} L_{\nu} \left( \textrm{erg s}^{-1}\textrm{Hz}^{-1} 
\right)$
  With Initial Mass Function (IMF) between $0.1 M_\odot$ 
  and $100 M_\odot$, in the range of $1250-2500 \mathring{\textrm{A}} $

The UV dust absorption \citep{kennicutt09} is not taken account in this work.

%Synthesis Models:
%1) Study individual star spectra, star atmospheres. Creation of spectral 
%libraries.
%2) ``Individual stellar templates are summed together weigthed by an initial 
%mass function (IMF)... These isochrones can the be added in linear combination 
%to syntesize the spectrum or colors of a galaxy with any arbitrary star 
%formation history, usually parametrized as an expinential function of time.
%*** A single model contains at least 4 parameters (SF history, galaxy age, 
%metal abundance, and IMF). The colors of normal galaxies are well represented 
%by a one-parameter sequence with fixed age, composition and IMF, varying in the 
%time dependence of the SFR****







\section{Observations}

This paper is based on three main observation sets. The most recent is from the
Hubble Space Telescope, this observations where performed by \cite{bouwens14}.
The second set is from the Canada-France Hawaii Telescope, \cite{willott13} and
the last one by the UK Infrared Telescope and the Subaru Telescope
\cite{mclure09}. Those observations where made using the Frop-out technique.

The data from the HST is a compilation of previous works since
2006 \citep{bouwens06}, wich includes also observations after the 2009
upgrade mission\citep{bouwens12}. 

The dataset was retrieved from graphs for \cite{bouwens14} and \cite{mclure09} 
using
GAVO-DEXTER\footnote{\url{http://dc.zah.uni-heidelberg.de/dexter/ui/ui/custom}}.



\begin{figure}
\epsscale{1.00} 
\plotone{fig/observational_data.pdf}
\caption{Observational data from \cite{bouwens14,mclure09}and \cite{willott13}.}
\label{graph_observational_data}
\end{figure}

     
\subsection{The Drop-out Technique - Lyman Break Technique}

\cite{steidel03}

\section{Discussion}

\begin{figure}
\epsscale{1.00}
\plotone{fig/cosmic_variance.pdf}
\caption{Cosmic Variance: The Luminosity Function is made using the DMH catalog 
from the full box and the set of parameters from the small boxes.}
\label{graph_cosmic_variance}
\end{figure}

Q: Mpc/h ?

\citep{lundgren14} SFR evolution from $z=1$ to $6$

\citep{bouwens14} HST Legacy

\citep{jiang11} Keck pectroscopy

\section{Summary}


\acknowledgments
%Gracias Totales
%
\appendix

\section{Appendix material}
DEXTER \url{http://dc.zah.uni-heidelberg.de/dexter/ui/ui/custom}

\begin{thebibliography}{}

\bibitem[Behroozi et al.(2013)]{behroozi13} Behroozi, Peter S., Risa H. Wechsler, 
\and Charlie Conroy. 2013, \apj,  770, 57
% Semi-Analytical Model for SFR-DMHM

\bibitem[Bouwens et al.(2006)]{bouwens06} Bouwens, R. J. et al. 2006, \apj, 653, 53	
		%2006ApJ...653...53B
% Observational luminosity function

\bibitem[Bouwens et al.(2012a)]{bouwens12} Bouwens, R. J. et al. 2012a, \apj, 752, 5 	
		%2012ApJ...752L...5B
% Observational luminosity function

\bibitem[Bouwens et al.(2012b)]{bouwens12b} Bouwens, R. J., G. D. Illingworth, 
P. A. Oesch, M. Franx, I. Labb�, M. Trenti, P. van Dokkum, et al. 2012b. \apj, 754,83
% Dust Attenuation 

\bibitem[Bouwens et al.(2014)]{bouwens14} Bouwens, R. J., G. D. Illingworth, P. A. 
Oesch, M. Trenti, I. Labbe', L. Bradley, M. Carollo, et al. 2014, arXiv:1403.4295
% Observational luminosity function

\bibitem[Col\'in et al.(1999)]{colin99}
Col\'in, Pedro, Anatoly A. Klypin, Andrey V. Kravtsov, and Alexei M. Khokhlov. 
1999, \apj , 523,32
% Halo Abundance Matching
\bibitem[Conroy et al.(2006)]{conroy06}
Conroy, Charlie, Risa H. Wechsler, \and Andrey V. Kravtsov. 2006, \apj,647,201
% Halo Abundance Matching

\bibitem[Efstathiou et al.(1988)]{efstathiou88} Efstathiou, G., Richard S. Ellis, 
\and Bruce A. Peterson. 1988, \mnras, 232,431.
% StepWise Maximun Likelihood LF

\bibitem[Finkelstein et al.(2014)]{finkelstein14} Finkelstein, Steven L., Russell E. 
Ryan Jr., Casey Papovich, Mark Dickinson, Mimi Song, Rachel Somerville, 
Henry C. Ferguson, et al. 2014, arXiv:1410.5439
% Observational luminosity function

\bibitem[Kennicutt(1998)]{kennicutt98} Kennicutt, Robert C., Jr. 1998, \araa, 
36, 189		%1998ARA&A..36..189K
\bibitem[Kennicutt et al.(2009)]{kennicutt09} Kennicutt, Robert C., Jr et al. 2009, \apj, 
703, 4672	%2009ApJ...703.1672K

\bibitem[Kravtsov et al.(2004)]{kravtsov04}
Kravtsov, Andrey V., Andreas A. Berlind, Risa H. Wechsler, Anatoly A. Klypin, 
Stefan Gottl�ber, Brandon Allgood, \and Joel R. Primack. 2004, \apj, 609, 35
%Halo Abundance Matching

\bibitem[Klypin et al.(2011)]{klypin11} Klypin, Anatoly A., Sebastian Trujillo-Gomez, \
and Joel Primack. 2009, \apj, 740, 102
% Bolshoi 2011

\bibitem[Klypin et al.(2014)]{klypin14} Anatoly Klypin, Gustavo Yepes, Stefan Gottlober, 
Francisco Prada \and Steffen Hess. 2014, arXiv:`1411.4001
% Big Bolshoi Planck 1 - MDPL

\bibitem[Law et al.(2011)]{law11} Law, K. et al. 2011, \apj, 738, 124
		%2011ApJ...738..124L
\bibitem[Jiang et al.(2011)]{jiang11} Jiang, Linhua et al. 2011, \apj, 743, 65
		%2011ApJ...743...65J
\bibitem[Lee et al.(2009)]{lee09} Lee, Kyoung-Soo et al. 2009, \apj, 695, 368
		%2009ApJ...695..368L
\bibitem[Lundgren et al.(2014]{lundgren14} Lundgren, Britt F. et al, 2014, \apj, 780,
34		%2014ApJ...780...34L
\bibitem[Madau et al.(1998)]{madau98} Madau, Piero. et al, 1998, \apj, 498, 106M
		%1998ApJ...498..106M
\bibitem[McLure et al.(2009)]{mclure09} McLure, R. J., M. Cirasuolo, J. S. Dunlop, S. Foucaud, 
\and O. Almaini. 2009, \mnras, 395, 2196	
		%2009MNRAS.395.2196M

\bibitem[Meurer et al.(1999)]{meurer99} Meurer, Gerhardt R., Timothy M. Heckman, 
\and Daniela Calzetti. 1999, \apj, 521, 64
% Dust Attenuation

\bibitem[Moster et al.(2010)]{moster10} Moster, Benjamin P. et al. 2010, \apj, 710, 
903 

\bibitem[Planck Collaboration et al(2014)]{planck1}
Planck Collaboration, P. A. R. Ade, N. Aghanim, C. Armitage-Caplan, M. Arnaud, 
M. Ashdown, F. Atrio-Barandela, et al. \aap, 571, A16
%Planck 2013 cosmology release.

\bibitem[Prada et al.(2012)]{prada12} Prada, Francisco, Anatoly A. Klypin, 
Antonio J. Cuesta, Juan E. Betancort-Rijo, and Joel Primack. 2012. \mnras, 423, 3018
% Big Bolshoi

\bibitem[Riebe et al.(2013)]{riebe13} Riebe, K. et al. 2013, AN, 334, 691 		
% Multiyear Database Release		%2013AN....334..691R

\bibitem[Schmidt(1968)]{schmidt68} Schmidt, Maarten. 1968, \apj 151, 393.
% StepWise Maximun Likelihood LF

\bibitem[Smit et al.(2012)]{smit12} Smit, Renske, Rychard J. Bouwens, Marijn Franx, Garth D. 
Illingworth, Ivo Labb�, Pascal A. Oesch, \and Pieter G. van Dokkum. 2012, \apj 756,14
% Dust Attenuation

\bibitem[Steidel et al.(2003)]{steidel03} Steidel, Charles C. et al. 2003, \apj, 592, 
728 		%2003ApJ...592..728S
\bibitem[Trimble(1987)]{trimble87} Trible, Virginia. 1987, \araa, 25, 425

\bibitem[van den Bosch et al.(203)]{vandenbosch03} van den Bosh, Frank C. et al. 2003, \mnras,
40, 771		%2003MNRAS.340..771V

\bibitem[Willott et al.(2013)]{willott13} Willott, Chris J., Ross J. McLure, Pascale Hibon,
 Richard Bielby, Henry J. McCracken, Jean-Paul Kneib, Olivier Ilbert, David G. Bonfield,
  Victoria A. Bruce, \and Matt J. Jarvis. 2013, \aj, 145, 4	
		%2013AJ....145....4W
\end{thebibliography}

\end{document}

%%
%% End of file `sample.tex'.
