\documentclass[manuscript]{aastex}
%\documentclass[preprint2]{aastex}
\newcommand{\vdag}{(v)^\dagger}
\newcommand{\myemail}{skywalker@galaxy.far.far.away}

\shorttitle{A Connection Between Star Formation Rate and Dark Matter Halos at Z \~ 6}
\shortauthors{G\'omez-Cort\'es}

\begin{document}

\title{A Connection Between Star Formation Rate and Dark Matter Halos at $Z\sim6$ Using 2013 Plank Cosmology.}

\author{F.L. G\'omez-Cort\'es\altaffilmark{1} }
\affil{Departamento de Física, Universidad de los Andes, Colombia}

\begin{abstract}
This work relates baryonic matter and dark matter at high redshift using observational data
from the HST and results of the Multidark Simulation.
Luminosity Function - Dark Matter Halos, Star Formation Rate - Dark Matter Halos.
\end{abstract}

\keywords{LF, SFR, high redshift galaxies, reionization}

\section{Introduction}
\subsection{Halo Mass Function (HMF)}
The HMF is created using the DMH catalog from the Multidark Database.
\subsection{Cosmic Variance}
Is important to know and understand the existence of local fluctuations
or inhomogeneities due the observed scale.

For this part is necesary to study the HMF on smaller boxes

\section{Relationship between Galaxy Luminosity Function (GLF) and
Dark Matter Halo (DMH) mass}

In one hand we have observational results: the GLF for star-forming galaxies at high redshift (z=5.9)\citep{bouwens06,willott07}
expressed in terms of magnitude in the ultraviolet range (M1350). In the other hand we have a DMH 
catalog, result from the Multidark Simulation with Plank Cosmology. %\citep{Multidark}

A $1000 \textrm{ Mpc} ^3 \textrm{h}^{−3}$ cubic box containing near to $11\times10^6$ dark matter halos.

The idea is to connect the observed GLF with the DMH catalog. We assume that each halo contains one and just one galaxy, and its luminosity is given by:
\begin{equation}
L_\textrm{galaxy}=\alpha M_\textrm{halo}^\beta 
\end{equation}
but this function is given in magnitude units. Is necesary to convert magnitude to luminosity.

\section{Magnitude to Luminosity}
The Luminosity is an intrinsic propertie of the stars. It doesn't deppends of the distance. 
It's directly related to the energy flux emmited. The luminosity $L_\nu$ at a given frecuency 
$\nu$ is has $[\textrm{W}\textrm{Hz}^{-1}]$ or $[\textrm{erg }\textrm{s}^{-1}\textrm{Hz}^{-1}]$ units.

The magnitude is brightnes star classification inherited from ancient Greeks. It deppends of 
the stellar distance. The absolute magnitude is a modern classification independent of 
distance. Taking the Sun as reference, the absolute magnitud at a given wavelengt is given by: 
\[ M_{\lambda} = M_{\lambda \odot} - 2.5 \log_{10}\left( \frac{L_\lambda}{L_{\lambda \odot}} \right) \]
The solar absolute magnitude in the U filter is:
\[ M_{\textrm{U} \odot} = 5.61\]
and the luminosity:
\[ 1 L_{\textrm{U} \odot} = 10^{18.48} \textrm{ergs s}^{-1}\textrm{Hz}^{-1}\]
\[ 1 L_{\textrm{U} \odot} = 3.02 \times 10^{18} \textrm{ergs s}^{-1}\textrm{Hz}^{-1}\]
Replacing in the absolute magnitude equation:
\[ M_{\textrm{U}} = 5.61 - 2.5 \log_{10}(L_{\textrm{U}}) + 25\times18.48\]
gives:
\[ M_{\textrm{U}} = 51.82 - 2.5 \log_{10}(L_{\textrm{U}}) \]

\subsection{Best Fitting Parameters}

\huge
A big mistake.

\[\log \neq \log_{10}\]
\normalsize

--Here comes the fitting graph--

And the fitting parameters

$\alpha = ?$

$\beta = ?$

\section{Stelar Formation Rate}

\subsection{The UV Continuum}
The UV region is dominated by big blue young stars, is possible to make an extrapolation to small young stars.

On stellar-formation galaxies the spectrum has a UV continuum nearly flat. This is a good approximation:

$ \textrm{SFR}\left(M_\odot \textrm{yr}^{-1}\right) = 1.4 \times 10^{28} L_{\nu} \left( \textrm{erg s}^{-1}\textrm{Hz}^{-1} \right)$

With Initial Mass Function (IMF) between $0.1 M_\odot$ and $100 M_\odot$, in the range of $1250-2500 \mathring{\textrm{A}} $


\subsection{The Fitting Model}

This fitting model contains four parameters: $\left(m/M\right)_0$, $M_1$, $\beta$ and $\gamma$, 
where $m:=$ Stellar mass, and $M :=$ Dark Matter Halo mass.
\begin{equation}
\frac{m}{M} = 2 \left( \frac{m}{M} \right)_{0} 
		  \left[ \left(\frac{M}{M_1}\right)^{-\beta} + \left(\frac{M}{M_1}\right)^{\gamma} \right]^{-1} 
\end{equation}
Another model cited in the article contains five parameters: $m_0$, $M_1$, $\beta$, $\gamma_1$ and $\gamma_2$.
\[ m(M) = m_0 \frac{ (M/M_1)^{\gamma_1}}{ \left[ 1 + (M/M_1)^\beta \right]^{ (\gamma_1-\gamma_2)/\beta}} \]


\section{Observations}

Willott

Bowens

McLure


\section{Discussion}

\section{Summary}


\acknowledgments
Gracias Totales


\appendix

\section{Appendix material}


\begin{thebibliography}{}
\bibitem[Bouwens(2006)]{bouwens06} Bouwens, R. J. et al. 2006, \apj, 653, 53
\bibitem[Willot(2007)]{willott07} Willott, Chris J. et al. 2013, \aj, 145, 4
\end{thebibliography}


\end{document}

%%
%% End of file `sample.tex'.
